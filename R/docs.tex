% Options for packages loaded elsewhere
\PassOptionsToPackage{unicode}{hyperref}
\PassOptionsToPackage{hyphens}{url}
%
\documentclass[
]{article}
\usepackage{lmodern}
\usepackage{amssymb,amsmath}
\usepackage{ifxetex,ifluatex}
\ifnum 0\ifxetex 1\fi\ifluatex 1\fi=0 % if pdftex
  \usepackage[T1]{fontenc}
  \usepackage[utf8]{inputenc}
  \usepackage{textcomp} % provide euro and other symbols
\else % if luatex or xetex
  \usepackage{unicode-math}
  \defaultfontfeatures{Scale=MatchLowercase}
  \defaultfontfeatures[\rmfamily]{Ligatures=TeX,Scale=1}
\fi
% Use upquote if available, for straight quotes in verbatim environments
\IfFileExists{upquote.sty}{\usepackage{upquote}}{}
\IfFileExists{microtype.sty}{% use microtype if available
  \usepackage[]{microtype}
  \UseMicrotypeSet[protrusion]{basicmath} % disable protrusion for tt fonts
}{}
\makeatletter
\@ifundefined{KOMAClassName}{% if non-KOMA class
  \IfFileExists{parskip.sty}{%
    \usepackage{parskip}
  }{% else
    \setlength{\parindent}{0pt}
    \setlength{\parskip}{6pt plus 2pt minus 1pt}}
}{% if KOMA class
  \KOMAoptions{parskip=half}}
\makeatother
\usepackage{xcolor}
\IfFileExists{xurl.sty}{\usepackage{xurl}}{} % add URL line breaks if available
\IfFileExists{bookmark.sty}{\usepackage{bookmark}}{\usepackage{hyperref}}
\hypersetup{
  pdftitle={docs.R},
  pdfauthor={chongtang},
  hidelinks,
  pdfcreator={LaTeX via pandoc}}
\urlstyle{same} % disable monospaced font for URLs
\usepackage[margin=1in]{geometry}
\usepackage{color}
\usepackage{fancyvrb}
\newcommand{\VerbBar}{|}
\newcommand{\VERB}{\Verb[commandchars=\\\{\}]}
\DefineVerbatimEnvironment{Highlighting}{Verbatim}{commandchars=\\\{\}}
% Add ',fontsize=\small' for more characters per line
\usepackage{framed}
\definecolor{shadecolor}{RGB}{248,248,248}
\newenvironment{Shaded}{\begin{snugshade}}{\end{snugshade}}
\newcommand{\AlertTok}[1]{\textcolor[rgb]{0.94,0.16,0.16}{#1}}
\newcommand{\AnnotationTok}[1]{\textcolor[rgb]{0.56,0.35,0.01}{\textbf{\textit{#1}}}}
\newcommand{\AttributeTok}[1]{\textcolor[rgb]{0.77,0.63,0.00}{#1}}
\newcommand{\BaseNTok}[1]{\textcolor[rgb]{0.00,0.00,0.81}{#1}}
\newcommand{\BuiltInTok}[1]{#1}
\newcommand{\CharTok}[1]{\textcolor[rgb]{0.31,0.60,0.02}{#1}}
\newcommand{\CommentTok}[1]{\textcolor[rgb]{0.56,0.35,0.01}{\textit{#1}}}
\newcommand{\CommentVarTok}[1]{\textcolor[rgb]{0.56,0.35,0.01}{\textbf{\textit{#1}}}}
\newcommand{\ConstantTok}[1]{\textcolor[rgb]{0.00,0.00,0.00}{#1}}
\newcommand{\ControlFlowTok}[1]{\textcolor[rgb]{0.13,0.29,0.53}{\textbf{#1}}}
\newcommand{\DataTypeTok}[1]{\textcolor[rgb]{0.13,0.29,0.53}{#1}}
\newcommand{\DecValTok}[1]{\textcolor[rgb]{0.00,0.00,0.81}{#1}}
\newcommand{\DocumentationTok}[1]{\textcolor[rgb]{0.56,0.35,0.01}{\textbf{\textit{#1}}}}
\newcommand{\ErrorTok}[1]{\textcolor[rgb]{0.64,0.00,0.00}{\textbf{#1}}}
\newcommand{\ExtensionTok}[1]{#1}
\newcommand{\FloatTok}[1]{\textcolor[rgb]{0.00,0.00,0.81}{#1}}
\newcommand{\FunctionTok}[1]{\textcolor[rgb]{0.00,0.00,0.00}{#1}}
\newcommand{\ImportTok}[1]{#1}
\newcommand{\InformationTok}[1]{\textcolor[rgb]{0.56,0.35,0.01}{\textbf{\textit{#1}}}}
\newcommand{\KeywordTok}[1]{\textcolor[rgb]{0.13,0.29,0.53}{\textbf{#1}}}
\newcommand{\NormalTok}[1]{#1}
\newcommand{\OperatorTok}[1]{\textcolor[rgb]{0.81,0.36,0.00}{\textbf{#1}}}
\newcommand{\OtherTok}[1]{\textcolor[rgb]{0.56,0.35,0.01}{#1}}
\newcommand{\PreprocessorTok}[1]{\textcolor[rgb]{0.56,0.35,0.01}{\textit{#1}}}
\newcommand{\RegionMarkerTok}[1]{#1}
\newcommand{\SpecialCharTok}[1]{\textcolor[rgb]{0.00,0.00,0.00}{#1}}
\newcommand{\SpecialStringTok}[1]{\textcolor[rgb]{0.31,0.60,0.02}{#1}}
\newcommand{\StringTok}[1]{\textcolor[rgb]{0.31,0.60,0.02}{#1}}
\newcommand{\VariableTok}[1]{\textcolor[rgb]{0.00,0.00,0.00}{#1}}
\newcommand{\VerbatimStringTok}[1]{\textcolor[rgb]{0.31,0.60,0.02}{#1}}
\newcommand{\WarningTok}[1]{\textcolor[rgb]{0.56,0.35,0.01}{\textbf{\textit{#1}}}}
\usepackage{graphicx,grffile}
\makeatletter
\def\maxwidth{\ifdim\Gin@nat@width>\linewidth\linewidth\else\Gin@nat@width\fi}
\def\maxheight{\ifdim\Gin@nat@height>\textheight\textheight\else\Gin@nat@height\fi}
\makeatother
% Scale images if necessary, so that they will not overflow the page
% margins by default, and it is still possible to overwrite the defaults
% using explicit options in \includegraphics[width, height, ...]{}
\setkeys{Gin}{width=\maxwidth,height=\maxheight,keepaspectratio}
% Set default figure placement to htbp
\makeatletter
\def\fps@figure{htbp}
\makeatother
\setlength{\emergencystretch}{3em} % prevent overfull lines
\providecommand{\tightlist}{%
  \setlength{\itemsep}{0pt}\setlength{\parskip}{0pt}}
\setcounter{secnumdepth}{-\maxdimen} % remove section numbering

\title{docs.R}
\author{chongtang}
\date{2020-01-22}

\begin{document}
\maketitle

@title SWATH-MS Gold Standard Dataset

@name Rost2014sgs

@description The SWATH-MS Gold Standard (SGS) dataset consists of 90
SWATH-MS runs of 422 synthetic stable isotope-labeled standard (SIS)
peptides in ten different dilution steps (1, 2, 4, 8, \ldots, 512
times), spiked into three protein backgrounds of varying complexity
(water, yeast and human), acquired in three technical replicates. The
SGS dataset was manually annotated, resulting in 342 identified and
quantified peptides with three or four transitions each. In total,
30,780 chromatograms were inspected and 18,785 were annotated with one
true peak group, whereas in 11,995 cases no peak was detected.

See also \url{http://www.openswath.org/openswath_data.html} for details.

@aliases Rost2014sgs Rost2014sgsHuman Rost2014sgsYeast Rost2014sgsWater

@docType data

@usage data(Rost2014sgsHuman) data(Rost2014sgsYeast)
data(Rost2014sgsWater)

@format The data is an instance of class \code{MSnSet} from package
\code{MSnbase}.

@details

@keywords datasets

@references
\emph{OpenSWATH enables automated, targeted analysis of data-independent
acquisition MS data.} Röst HL, Rosenberger G, Navarro P, Gillet L,
Miladinović SM, Schubert OT, Wolski W, Collins BC, Malmström J,
Malmström L, Aebersold R. Nat Biotechnol. 2014 Mar;32 (3):219-23. doi:
10.1038/nbt.2841.
(\href{https://www.ncbi.nlm.nih.gov/pubmed/23979570}{PubMed})

@source
\href{http://compms.org/resources/reference-data/50}{SWATH-MS Gold Standard Dataset}

@examples library(msdata2) data(Rost2014sgsHuman) Rost2014sgsHuman
pData(Rost2014sgsHuman) head(exprs(Rost2014sgsHuman))

\begin{Shaded}
\begin{Highlighting}[]
\OtherTok{NULL}
\end{Highlighting}
\end{Shaded}

\begin{verbatim}
## NULL
\end{verbatim}

@title Spiked Proteomic Standard Datasets For Benchmark 8 Different
label-free Quantitative Workflows

@name ramus2016

@description This data consists of quantitative results from 8 different
Label-Free quantitative workflows for a controlled, spiked proteomic
standard dataset. The proteomics standard data was prepared using a
yeast cell lysate accompanied by a serial dilution (respectively
0.05-0.125-0.250-0.5-2.5-5- 12.5-25-50 fmol of UPS1 /\(\mu\)g of yeast
lysate) of the UPS1 standard mixture (Sigma). Samples were then analyzed
in triplicate by nanoLC-MS/MS coupled to an LTQ-Orbitrap Velos mass
spectrometer, operated in data-dependent acquisition mode.

The dataset was then processed by 8 different workflows for
benchmarking, consisting in the following steps: peaklist generation,
database search, validation of the identified proteins and extraction of
quantitative metric (spectral count or MS signal).

@aliases ramus2016 ramus2016SCMfPaQ ramus2016SCMaxQuant
ramus2016SCIrmaHeidi ramus2016SCScaffold ramus2016IntMFPaQ
ramus2016IntMaxQuant ramus2016LFQMaxQuant ramus2016IntSkyline

@docType data

@usage data(ramus2016SCMfPaQ) data(ramus2016SCMaxQuant)
data(ramus2016SCIrmaHeidi) data(ramus2016SCScaffold)
data(ramus2016IntMFPaQ) data(ramus2016IntMaxQuant)
data(ramus2016LFQMaxQuant) data(ramus2016IntSkyline)

@format The data is an instance of class \code{MSnSet} from package
\code{MSnbase}.

@details Full details of the experimental design can be found in the
reference. The 8 different datasets correspond to 8 different LFQ
workflows: \texttt{ramus2016SCMfPaQ}: Peaklist creation device -
ExtractMSn Database search engine - Mascot Validation/Spectral counting
device - MFPaQ Quantification method - Spectral counting

@keywords datasets

@references
\emph{OpenSWATH enables automated, targeted analysis of data-independent
acquisition MS data.} Röst HL, Rosenberger G, Navarro P, Gillet L,
Miladinović SM, Schubert OT, Wolski W, Collins BC, Malmström J,
Malmström L, Aebersold R. Nat Biotechnol. 2014 Mar;32 (3):219-23. doi:
10.1038/nbt.2841.
(\href{https://www.ncbi.nlm.nih.gov/pubmed/23979570}{PubMed})

@source
\href{http://compms.org/resources/reference-data/50}{SWATH-MS Gold Standard Dataset}

@examples library(msdata2) data(ramus2016IntSkyline) Rost2014sgsHuman
pData(ramus2016IntSkyline) head(exprs(ramus2016IntSkyline))

\begin{Shaded}
\begin{Highlighting}[]
\OtherTok{NULL}
\end{Highlighting}
\end{Shaded}

\begin{verbatim}
## NULL
\end{verbatim}

\end{document}
